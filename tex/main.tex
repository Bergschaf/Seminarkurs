\documentclass{article}

\begin{document}

\section{Alkoholkonsum in Baden Württemberg}



\subsection{Im zeitlichen Verlauf}
Schon im alten Griechenland wird in vielen Erzählungen über Saufgelage berichtet (TODO Quelle), damals war der Alkohol im Alltag allerdings verpönt (TODO Quelle). Erst seit dem Mittelarlter spielt Alkohol eine große Rolle in Europa. Ab dem ?? Jahrhundert wird er als ein wichtiger Teil der Ernährung von allen Altersgruppen mit etwa 3 Litern täglich Konsumiert (TODO Quelle). Man muss aber beachten, dass die Art dies Alkohols, die damals konsumiert wurde stark von heute abweicht. Ein großer Teil des Alkohols wurde z.B. in Form von Biersuppen konsumiert (TODO Quelle), bei deren Kochvorgang allerdings der Alkoholgehalt stark sank (TODO Quelle). Zu Beginn des 17. Jahrhunderts verbreiteten sich neue Genussmittel wie Kaffee, Kakao und Zucker in Europa, was zu einem starken Abfall des Alkoholkonsums führte (TODO Quelle). 

\subsection{Im vergleich zu anderen Bundesländern}
Viele würden es als gegeben ansehen, dass zwischen dem Trinkverhalten in den nördlichen und den südlichen Bundesländern in Deutschland ein großes Gefälle besteht. Ein Beispiel dafür ist zum Beispiel das bayrische Oktoberfest, bei dem enorme Mengen an Alkohol konsumiert werden. Es wird auch als das größte Drogenfestival in Deutschland bezeichnet (TODO Quelle). Die Wissenschaft ist sich allerdings uneinig ob dieses Vorurteil wirdklich zutrifft. Es gibt Studien die behaupten, dass es durchaus ein Gefälle zwischen dem Trinkverhalten im Norden und im Süden Deutschlands gibt (TODO Quelle @Regionale Unterschiede in der Prävalenz riskanten Alkoholkonsums: Sekundäranalyse des Gesundheitssurveys Ost-West). Eine andere Studie behauptet wiederum, dass diese Differenzen nur durch falsche Datenerhebung hervorgerufen werden. Um diese Behauptungen genauer zu analysieren haben wir die Methoden dieser Studien untersucht: 
\subsubsection {....}
Dafür müssen wir zuerst definieren, welche Bundesländer zu Nord- und Süd- Deutschland gehörren.... 
Metrik mit der gemessen wird: Menge an konsumiertem Reinalkohol (pro Trinktag ??)
....
Ergebniss: Es gibt keine Messbare differenz in der Menge an konsumierten reinalkohol zwischen Nord und Süddeutschland
---

Wenn man allerdings die einzelnen Bundesländer genauer betrachtet fällt auf, dass in Bayern tatsächlich überdurchschnittlich viel Alkohol konsumiert wird (TODO zitate und tabelle aus: @Der Einfluss regionaler Unterschiede im Trinkstil auf riskanten Konsum, exzessives Trinken, Missbrauch und Abhängigkeit). In Baden Württemberg wird allerdings durchschnittlich relativ wenig Alkohol pro Tag getrunken. Auch der riskante Alkoholkonsum ist relativ gering. (Verlgeich mit eigener Krankenhausstatistitk). 


\end{document}
