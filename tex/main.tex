\documentclass{article}

\begin{document}

\section{Alkoholkonsum in Baden Württemberg}



\subsection{Im zeitlichen Verlauf}
Schon im alten Griechenland wird in vielen Erzählungen über Saufgelage berichtet (TODO Quelle), damals war der Alkohol im Alltag allerdings verpönt (TODO Quelle). Erst seit dem Mittelarlter spielt Alkohol eine große Rolle in Europa. Ab dem ?? Jahrhundert wird er als ein wichtiger Teil der Ernährung von allen Altersgruppen mit etwa 3 Litern täglich Konsumiert (TODO Quelle). Man muss aber beachten, dass die Art dies Alkohols, die damals konsumiert wurde stark von heute abweicht. Ein großer Teil des Alkohols wurde z.B. in Form von Biersuppen konsumiert (TODO Quelle), bei deren Kochvorgang allerdings der Alkoholgehalt stark sank (TODO Quelle). Zu Beginn des 17. Jahrhunderts verbreiteten sich neue Genussmittel wie Kaffee, Kakao und Zucker in Europa, was zu einem starken Abfall des Alkoholkonsums führte (TODO Quelle). 

\subsection{Im vergleich zu anderen Bundesländern}


\end{document}
