\documentclass[12pt]{article}


\PassOptionsToPackage{hyphens}{url}\usepackage{hyperref}\usepackage{breakurl}
\usepackage{datetime}
\usepackage{enumitem}
\usepackage[german]{babel}
\usepackage{tocloft}

\setlength\cftparskip{2pt}
\setlength\cftbeforesecskip{1pt}
\setlength\cftaftertoctitleskip{2pt}

\setitemize{itemsep=1pt,topsep=1pt}
\setcounter{secnumdepth}{0}

\usepackage{titlesec}
\usepackage{lipsum}% just to generate text for the example

\titlespacing*{\section}
{0pt}{1.5ex plus 1ex minus .2ex}{2.3ex plus .2ex}
\titlespacing*{\subsection}
{0pt}{5.5ex plus 1ex minus .2ex}{4.3ex plus .2ex}


%\setlist{nosep}
\usepackage[a4paper,top=2cm,bottom=2cm,left=2.5cm,right=2.5cm,marginparwidth=1.75cm]{geometry}

\begin{document}
\title{Alkohol}
    \begin{center}
        %\vspace*{1cm}
        
        {\huge{Alkohol}}
 
        \vspace{0.7cm}
         {\Large Eine Gefahr für die Jugend in Baden-Württemberg?}
             
        \vspace{0.5cm}
 
        \textbf{Lukas Vogel \& Christian Krause}
        \vspace{1cm}
             
        %\vspace{cm}
        %\vfill
        %\includegraphics[width=0.2\textwidth]{assets/gymox.png}
        %2\large
                  
    \end{center}

 \tableofcontents
    \vspace{1cm}
 \section{Gefahren}
 \begin{itemize}
    \item Sucht: psychisch und physisch
    \item Gesellschaftlich: Verlust von sozialen Verhaltensweisen
    \item Krebs: In Europa sind über 200 Mio. Menschen bedroht, durch Alkoholkonsum an Krebs zu erkranken
    \item Geschlechtsspezifisch
        \begin{itemize}[nosep]
            \item Frauen: Folgen für das Kind bei einer Schwangerschaft
            \item Männer: Störungen des Denkens im Alter
        \end{itemize}
    \item Jugendliche: Fehlbildungen im Hippocampus
 \end{itemize}

 \section{Ursachen des Konsums}
 \begin{itemize}
    \item Buden: Alkohol wird auch an Jugendliche unter dem Mindestalter in großen Mengen ausgeschenkt 
    \item Eltern: Manche Eltern haben eine sehr unkritische Einstellung zu Alkohol
 \end{itemize}

 \section{Daten}
 \begin{itemize}
    \item Die Trinkmenge und der riskante Konsum sind in den nördlichen und südlichen Bundesländern ungefähr gleich
    \item Jährlich werden deutschlandweit durchschnittlich ca. 26 Jugendliche pro 100.000 Einwohner mit einer Alkoholintoxikation im Krankenhaus behandelt
    \item Baden-Württemberg liegt in dieser Statistik leicht über dem Bundesdurchschnitt, Saarland und Bayern sind an der Spitze
 \end{itemize}
 
 \section{Problem}
 \begin{itemize}
    \item Der Alkoholkonsum ist in Deutschland mehr als doppelt so hoch wie der globale Durchschnitt
    \item Es herrscht eine sehr unkritische Einstellung gegenüber Alkohol in der Gesellschaft
    \item Viele Jugendliche müssen mit Alkoholintoxikation im Krankenhaus behandelt werden
 \end{itemize}
 \par$\Rightarrow$ Alkohol ist durchaus eine Gefahr für die Jugend in Baden-Württemberg

 \section{Präventionsmaßnahmen}
 \begin{itemize}
    \item Verbote sind oft nicht effektiv, da sie leicht umgangen werden können und wenig Rückhalt in der Gesellschaft haben
    \item Strukturelle Präventionsmaßnahmen verringern Angebot und Nachfrage von Alkohol, z.B. durch höhere Steuern oder örtliche und zeitliche Verkaufseinschränkungen $\rightarrow$ sehr effektiv
    \item Individuelle Präventionsmaßnahmen versuchen, die Ursachen für Alkoholkonsum bei Risikoschülern durch personalisierte Interventionen zu bekämpfen $\rightarrow$ auch bei Nicht-Risikoschülern effektiv
\end{itemize}
\vspace{0.5cm}
\section{Quellen}
\footnotesize

\begin{itemize}
    \item \url{https://www.oktoberfest.de/informationen/termine/eroeffnung-und-anstich-des-oktoberfests}
    \item \url{https://www.schwaebische.de/regional/biberach/laupheim/zehn-von-zwoelf-laupheimer-geschaefte-verkaufen-alkohol-an-jugendliche-testperson-1834461}
    \item \url{https://www.schwaebische.de/regional/biberach/biberach/schuetzenfest-polizei-erwischt-jugendliche-mit-alkohol-und-haschisch-772222}
    \item \url{https://www.tagesschau.de/wissen/who-alkoholkonsum-100.html}
    \item \url{https://www.ju-bib.de/de/Buden}
    \item \url{https://www.schwaebische.de/regional/biberach/biberach/das-wichtigste-zum-biberacher-schuetzenfest-2024-im-uberblick-2589533}
    \item \url{https://www.futura-sciences.com/de/wp-content/uploads/2022/04/alkohol-und-alkoholische-getranke-historisches-und-wissenswertes.jpg}
    \item Kraus, Ludwig, R. Augustin, Kim Bloomfield und A. Reese. “Der Einfluss regionaler Unterschiede im Trinkstil auf riskanten Konsum, exzessives Trinken, Missbrauch und Abhängigkeit”. Gesundheitswesen 63 (1. Dezember 2001)
\end{itemize}
\begin{center}
    \vspace{2cm}
    Seminarkurs von Frau Titze 2023/2024\\
    Gymnasium Ochsenhausen\\
    Herrschaftsbrühl 12, 88416 Ochsenhausen\\
    \newdate{date}{27}{06}{2024}
    \displaydate{date}

\end{center}

\end{document}