\documentclass[12pt]{article}

\usepackage[german]{babel}
\usepackage{datetime}
\usepackage{enumitem}
\setenumerate{label=(\roman*),itemsep=2pt,topsep=2pt}
%\setlist{nosep}
\usepackage[a4paper,top=2cm,bottom=2cm,left=3cm,right=4cm,marginparwidth=1.75cm]{geometry}

\begin{document}
\title{Alkohol}
\begin{titlepage}
    \begin{center}
        %\vspace*{1cm}
        
        {\huge{Alkohol}}
 
        \vspace{0.7cm}
         {\Large Eine Gefahr für die Jugend in Baden-Württemberg?}
             
        \vspace{0.5cm}
 
        \textbf{Lukas Vogel \& Christian Krause}
        \vspace{1cm}
             
        %\vspace{cm}
        %\vfill
        %\includegraphics[width=0.2\textwidth]{assets/gymox.png}
        \contentsline {section}{\numberline {1}Einleitung}{2}{section.1}%
\contentsline {section}{\numberline {2}Gefahren \footnotesize {- Lukas Vogel}}{2}{section.2}%
\contentsline {subsection}{\numberline {2.1}Sucht}{3}{subsection.2.1}%
\contentsline {subsection}{\numberline {2.2}Gesellschaftliche Folgen}{4}{subsection.2.2}%
\contentsline {subsection}{\numberline {2.3}Krebs}{5}{subsection.2.3}%
\contentsline {subsection}{\numberline {2.4}Geschlechtsspezifische Auswirkungen}{5}{subsection.2.4}%
\contentsline {subsection}{\numberline {2.5}Jugendliche}{6}{subsection.2.5}%
\contentsline {subsection}{\numberline {2.6}Fazit}{7}{subsection.2.6}%
\contentsline {section}{\numberline {3}Ursachen des Konsums \footnotesize {- Lukas Vogel}}{7}{section.3}%
\contentsline {subsection}{\numberline {3.1}Buden}{7}{subsection.3.1}%
\contentsline {subsection}{\numberline {3.2}Eltern}{8}{subsection.3.2}%
\contentsline {section}{\numberline {4}Alkoholkonsum in Baden-Württemberg \footnotesize {- Christian Krause}}{8}{section.4}%
\contentsline {subsection}{\numberline {4.1}Vergleich zu anderen Bundesländern}{8}{subsection.4.1}%
\contentsline {subsection}{\numberline {4.2}Analyse der Krankenhausstatistik}{10}{subsection.4.2}%
\contentsline {section}{\numberline {5}Eigene Umfrage \footnotesize {- Lukas Vogel}}{14}{section.5}%
\contentsline {subsection}{\numberline {5.1}Auswertung}{14}{subsection.5.1}%
\contentsline {subsection}{\numberline {5.2}Fazit}{17}{subsection.5.2}%
\contentsline {section}{\numberline {6}Problem \footnotesize {- Christian Krause}}{18}{section.6}%
\contentsline {section}{\numberline {7}Prävention \footnotesize {- Christian Krause}}{20}{section.7}%
\contentsline {subsection}{\numberline {7.1}Empfehlungen des Gesundheitsministeriums}{20}{subsection.7.1}%
\contentsline {subsection}{\numberline {7.2}Verbote}{20}{subsection.7.2}%
\contentsline {subsection}{\numberline {7.3}Strukturelle Prävention}{21}{subsection.7.3}%
\contentsline {subsection}{\numberline {7.4}Individuelle Prävention}{22}{subsection.7.4}%
\contentsline {section}{\numberline {8}Schlusswort}{24}{section.8}%
\contentsline {section}{\numberline {9}Anhang}{25}{section.9}%
\contentsline {subsection}{\numberline {9.1}Literaturverzeichnis}{26}{subsection.9.1}%
\contentsline {subsection}{\numberline {9.2}Abbildungsverzeichnis}{28}{subsection.9.2}%
        \large
        \vspace{2cm}
        Seminarkurs von Frau Titze 2023/2024\\
        Gymnasium Ochsenhausen\\
        Herrschaftsbrühl 12, 88416 Ochsenhausen\\
        \newdate{date}{13}{05}{2024}
        \displaydate{date}
             
    \end{center}
 \end{titlepage}

 \clearpage

 \section*{Gefahren}
 \begin{itemize}
    \item 
 \end{itemize}

 \section*{Ursachen des Konsums}

 \section*{Daten}
 \begin{itemize}
    \item Die Trinkmenge und der riskante Konsum ist in den nördlichen und südlichen Bundesländern ungefähr gleich
    \item Jährlich werden Deutschlandweit durchschnittlich ca. 26 Jugendliche pro 100.000 Einwohner mit einer Alkoholintoxikation im Krankenhaus behandelt
    \item Baden-Württemberg liegt in dieser Statistik leicht über dem Bundesdurchschnitt, Saarland und Bayern sind an der Spitze
 \end{itemize}
 
 \section*{Ist Alkohol ein Problem für die Jugend?}
 \begin{itemize}
    \item Der Alkoholkonsum ist in Deutschland mehr als doppelt so hoch wie der globale Durchschnitt
    \item Es herrscht eine sehr unkritische Einstellung gegenüber Alkohol in der Gesellschaft
    \item Viele Jugendliche müssen mit Alkoholintoxikation im Krankenhaus behandelt werden
 \end{itemize}
 \par$\Rightarrow$ Alkohol ist durchaus eine Gefahr für die Jugend in Baden-Württemberg

 \section*{Präventionsmaßnahmen}
 \begin{itemize}
    \item Verbote sind oft nicht effektiv, da sie leicht umgangen werden können und wenig Rückhalt in der Gesellschaft haben
    \item Strukturelle Präventionsmaßnahmen verringern Angebot und Nachfrage von Alkohol z.B. durch höhere Steuern oder örtliche und zeitliche Verkaufseinschränkungen $\rightarrow$ sehr effektiv
    \item Individuelle Präventionsmaßnahmen versuchen die Ursachen für Alkoholkonsum bei Risikoschülern durch personalisierte Interventionen zu bekämpfen $\rightarrow$ auch bei Nicht-Risikoschülern effektiv
\end{itemize}

 \section*{Quellen}

\end{document}